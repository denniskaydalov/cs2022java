\documentclass{article}
\usepackage[letterpaper,top=2cm,bottom=2cm,left=0.6cm,right=0cm,marginparwidth=1.75cm]{geometry}

\usepackage{cpgf-umlcd}
\usepackage{tikz}

\begin{document}
\begin{tikzpicture}
    \begin{abstractclass}[text width = 11cm]{FillableShape}{0, 0}
        \attribute {- x1 : int}
        \attribute {- x2 : int}
        \attribute {- x3 : int}
        \attribute {- x4 : int}
        \attribute {- filled : boolean}

        \operation {+ FillabeShape(x1 : int, x2 : int, x3 : int, x4 : int, filled : boolean)}
        \operation {+ getUpperLeftX() : int}
        \operation {+ getUpperLeftY() : int}
        \operation {+ getWidth() : int}
        \operation {+ getHeight() : int}
        \operation {+ getX1()}
        \operation {+ getX2()}
        \operation {+ getX3()}
        \operation {+ getX4()}
        \operation {+ getFilled()}
        \operation {+ setX1(x1 : int)}
        \operation {+ setX2(x2 : int)}
        \operation {+ setX3(x3 : int)}
        \operation {+ setX4(x4 : int)}
        \operation {+ setFilled(filled : boolean)}
        \operation {+ toString() : String}
        \operation [0]{+ calcArea() : double}
    \end{abstractclass}

    \begin{class}[text width = 8cm]{Oval}{-6, -15}
        \inherit{FillableShape}
        \attribute {\underline{- ovalCount : int}}

        \operation {+ Oval(x1 : int, x2 : int, x3 : int, x4 : int, filled : boolean)}
        \operation {+ Oval()}
        \operation {+ isCircle() : boolean}
        \operation {+ calcArea() : double}
        \operation {\underline{+ getOvalCount() : int}}
    \end{class}

    \begin{class}[text width = 8cm]{Rectangle}{5, -15}
        \inherit{FillableShape}
        \attribute {\underline{- rectangleCount : int}}

        \operation {+ Rectangle(x1 : int, x2 : int, x3 : int, x4 : int, filled : boolean)}
        \operation {+ Rectangle()}
        \operation {+ isOverlapping() : boolean}
        \operation {+ calcArea() : double}
        \operation {\underline{+ getRectangleCount() : int}}
    \end{class}

\end{tikzpicture}
\end{document}
